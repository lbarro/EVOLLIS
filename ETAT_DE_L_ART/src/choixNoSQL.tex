Comme je l'ai mentionné dans le coxtexte à la section \ref{contexte}, Evollis utilise actuellement une
BDDR classique, PGSQL. Il utilise également un système de cache, le \textsf{Memcached}.Un peu plus au dessus
du present document à la section \ref{membase}, j'ai aussi parlé de Membase qui peut être vue comme une amélioration
de memcached, en ce sens qu'il est compatible avec toutes les applications memcached avec des fonctionnalités
supplémentaires telles que la persistance, le réplication et le clustering.  
Avec le système clé-valeur simple, le système n'est pas conscient de la structure de la valeur.
Evollis veut tester une autre solution NoSQL qui est MongoDB. MongoDB utilise la techno clé-valeur
mais rend le système conscient de la structure de la valeur stockée dans le but de permettre un champ
opérationnelle plus large sur la base de données.

En raisons du nombre de solutions \textsf{NoSQL} existants, il fallait opérer un choix. Je m'intéresserai donc uniquement
aux deux solutions membase et MongoDB pour les mêmes raisons mentionnées précédemment.  
