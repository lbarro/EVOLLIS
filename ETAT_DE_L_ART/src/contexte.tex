Sur un site de vente en ligne partenaire d'EVOLLIS, si un client forme un panier et décide de le payer, par le biais d'une vignette EVOLLIS sur laquelle le client pourra ensuite cliquer pour profiter de l'offre \textsf{lizéa}. Le calcul des tarifs de l'offre est effectué  à partir des identifiants des produits dans le panier du client et    


Evollis se place entre plusieurs partenaires commerciaux afin de 
coordonner leurs interactions.


Les cash sont volatiles, il faut à chaque les réconstitués. Les cash
fonctionnent en clé-valeur comme la plupart des gestionnaires NoSQL.
EVOLLIS veut en lieu et place du système de cash mettre un système de
hashage persistant apportant les mêmes performances qu'un accès cash.
Le Benefice étant que cette fois on a un systant persistant. À priori
le s'est pausé sur MongoDB. En arrière plan MongoDB reste une base
clef/valeur où la valeur est un document JSON complet. Cependant,
contrairement aux bases purement clef/valeur, il est possible de
requêter sur n’importe quel élément du document ou sous document.
