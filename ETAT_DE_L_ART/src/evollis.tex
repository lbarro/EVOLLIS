\section{La fiche d'identité}

\textsf{EVOLLIS} est une \textsf{SAS} - Société par Actions Simplifiée - au capital 
de $245 000$ € crée par \textsf{M. Xavier PINSE} en mars 2011. Sise à Bordeaux, 
\textsf{EVOLLIS} compte un effectif de $6$ personnes. Le fonctionnement 
d'\textsf{EVOLLIS} est essentiellement orienté vers de la sous-traitance. 
\textsf{EVOLLIS} propose un nouveau produit, \textsf{uZ'it}, qui est une 
nouvelle solution financière. Grâce à sa solution, \textsf{EVOLLIS} permet 
aux particuliers d'accéder à des produits haut de gamme et dernier cri,
pour quelques euros par mois sans craindre leur obsolescence.

\section{Présentation du produit \textsf{uZ'it}}

Le produit \textsf{uZ'it} est basé sur le principe de location avec option d'achat qui répond à 
un nouveau mode de consommation. Il est « packagée » et se compose de 3 services intégrés:
\\\\
\textsf{Le financement du produit en location}: une véritable innovation commerciale,
une alternative au crédit ou le client ne paye que la valeur d'usage et surtout la 
possibilité d'accéder à un produit d'une gamme supérieure grâce aux mensualités réduites.
\\\\
\textsf{La couverture complète durant toute la durée d'utilisation}: à la solution de 
financement par location, s'ajoute des services associés comme la garantie totale du
produit et le service Après-Vente. Le consommateur est ainsi est ainsi totalement
couvert durant toute la durée du contrat. 
\\\\
\textsf{La flexibilité en fin de contrat}: au terme du contrat, le consommateur est 
libre de choisir entre la restitution du produit, l'option d'achat ou l'acquisition
d'un nouveau bien «dernier cri».

\section{Les partenaires d'\textsf{EVOLLIS}}
\textsf{EVOLLIS} ne fonctionne pas tout seul, il se met en intermédiaire de plusieurs 
partenaires. Il combine ainsi les services de ses différents partenaires pour en faire 
un seul produit. Les partenaires d'\textsf{EVOLLIS} sont:
\begin{dinglist}{70}
\item Les distributeurs: en partenariat avec \textsf{EVOLLIS}, ils vont proposer le
produit \textsf{uZ'it} aux clients finaux.
\item Les organismes de financement qui financent les contrats de location.
\item Les structures de dématérialisations qui permettent de vérifier la validité des pièces 
justificatives fournies par le client final.
\item Les recycleurs qui interviennent à la récupération des produits à la fin du contrat
de location. 
\end{dinglist}

 
