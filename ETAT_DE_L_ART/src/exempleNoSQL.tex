Comme je l'ai mentionné à la section précédente, le
site \url{http://nosql-database.org} recense $122$
solutions \textsf{NoSQL}.  J'en présenterai seulement 5. Un exemple
par catégorie énumérée précédemment. Je m'intéresserai
particulièrement aux solutions « \textsf{Open Source} » très
utilisées. Je parlerai aussi de la solution propriétaire \textsf{BigTable}
qui est considérée comme la première de la
mouvance \textsf{NoSQL}. Dans l'ordre, \textsf{Membase} une solution
orientée \textsf{clé-valeur}, \textsf{MongoDB} une solution
orientée \textsf{document}, \textsf{cassandra} une solution
orientée \textsf{colonne}, \textsf{Néo4j} une solution
orientée \textsf{graphe} et \textsf{BigTable} une solution
propriétaire orientée \textsf{colonne}.
\\
\\
\textsf{Membase}: \index{Membase}\textsf{Membase} Solution \textsf{NoSQL} orientée \textsf{clé-valeur}, Open Source et diffusée sous la licence Apache 2.0. Écrit en \textsf{C++/Erlang}, il est soutenu par l'entreprise du même nom \textsf{Membase}\cite{RickCattell}. Il a été développé par les leaders du projet \textsf{Memcached} qui est un
autre système de stockage \textsf{clé-valeur}. Ce dernier permet la
gestion de mémoire cache distribuée, utilise la \textsf{RAM} et n'est
donc pas persistant. \textsf{Memcached} fut à l'origine développé
par \textsf{Brad Fitzpatrick} pour le \textsf{LiveJournal} en $2003$
et est aujourd'hui utilisé part de nombreux sites
tels \textsf{Wikipedia, Flickr, Bebo, Twitter, Typepad, Yellowbot,
Youtube, Digg, WordPress.com, Craigslist,
Mixi}\cite{memcached}. \textsf{Membase}
rajoute au système \textsf{Memcached} la propriété de persistance sur
le disque, la réplication des données pour assurer la résistance aux
pannes et la scalabilité horizontale. Il est compatible avec les applications memcached existantes.
\\
\\ 
{\sf MongoDB}: \index{MongoDB} Solution \textsf{NoSQL} orientée \textsf{document},
Open Source et diffusée sous la licence \textsf{GPL}. Écrit
en \textsf{C++}, il est soutenu
par \textsf{10gen}\cite{RickCattell}. Son développement a débuté en Octobre $2007$ et sa prémière version public est sortie en Février $2009$\cite{blogmongodb}. Sur son
site \url{http://www.10gen.com/what-is-mongodb}, \textsf{10gen} écrit
que le but de \textsf{MongoDB} est de combiner le
modèle \textsf{clé-valeur}, rapide et scalable, et le modèle
relationnel qui permet des opérations complexes comme les jointures et
l'indexation. \textsf{MongoDB} stocke les données sous le format
binaire \textsf{BSON} de \textsf{JSON}\footnote{Ces notions sont
expliquées plus en détail à la section \ref{mongodb} du présent
document} dont la structure reste libre et dynamique.  En effet, aucun
schéma de \textsf{BDD} à respecter n'est prédéfini\cite{mongoDB}. Le
site web «\textsf{ChinaVisual}», la plus grande média en ligne
chinoise
%\footnote{D'après \url{http://www.aboutus.org/ChinaVisual.com}}
,
a annoncé en début d'année $2009$ sa migration de \textsf{MySQL}
vers \textsf{MongoDB}\cite{GUYunhua} alors que \textsf{10gen} venait juste de publier la prémière version stable. Aujourd'hui, \textsf{MongoDB} est utilisé par de grands acteurs de la toile comme \textsf{Buddy Media, Craigslist, Disney, Forbes, foursquare, Intuit, MTV Networks, Shutterfly, Traackr, Wordnik}\cite{10genClients}.
\\
\\ 
\textsf{Cassandra}: \index{Cassandra} Solution \textsf{NoSQL} orientée 
\textsf{colonnes} de la fondation \textsf{Apache}, Open Source et diffusée sous la licence
 Apache 2.0. Comme toutes les solutions \textsf{NoSQL}, \textsf{Cassandra} 
met en avant le clustering. Il est écrit en 
\textsf{Java}\cite{RickCattell}. Initialement développé par 
\textsf{Facebook}, le code source est devenu « \textsf{Open-Source} » sur 
\textsf{Google code} en juillet $2008$: 
\url{http://code.google.com/p/the-cassandra-project/}
%\footnote{\url{http://perspectives.mvdirona.com/2008/07/12/FacebookReleasesCassandraAsOpenSource.aspx}}. 
En
mars $2009$, \textsf{Cassandra} rentre dans l'incubateur de projets
de la fondation \textsf{Apache}\index{Apache Incubator}, une passerelle de
validation de projets désireux d'intégrer la fondation. En Février
$2010$, \textsf{Cassandra} dévient un projet à part entière de la
fondation
%\footnote{\url{http://www.mail-archive.com/cassandra-dev@incubator.apache.org/msg01518.html}}. 
\textsf{Facebook}
scale \textsf{Cassandra} sur plus de $150$ machines. D'autres grands
sites tels que \textsf{Twitter et SoftwareProjects}
utilisent \textsf{Cassandra}. \textsf{SoftwareProjects} utilise 20
nœuds \textsf{Cassandra} à travers 3 entrepôts de données pour
alimenter leur plate-forme e-commerce électronique pour $3000$
entreprises. Il utilise \textsf{Cassandra} pour stocker des données
d'achat en temps réel et fournir des statistiques aux divers
clients. \textsf{Cassandra} est leur magasin de données
principal\cite{apacheClients}.
\\
\\
\textsf{Neo4j}: \index{Néo4j} Solution \textsf{NoSQL} orientée 
\textsf{gaphe} de l'entreprise \textsf{Neo Technology}, une entreprise suédoise\cite{MichaelFiguiereNeo4j}. Il est écrit en java\cite{GavinTerrill}. La Première version, \textsf{Neo4j 1.0} est sortie en Février $2010$\cite{Neo4jBlog}. 
Sur le site \url{http://neo4j.org/}, j'ai pu lire que \textsf{Neo4j}
est \textsf{Open Source} sous licence \textsf{AGPLv3} et offre des
performances $1000$ fois supérieures aux \textsf{BDDR} classiques. En
effet \textsf{Neo4j} permet à une application de bénéficier de toute
l'expressivité des graphes pour modéliser toutes les dépendances non
triviales entre les données. Il permet aussi de bénéficier
d'algorithmes très optimisées de recherche et d'ajout d'éléments dans
un graphe. L'utilisation de graphe permet surtout de répondre
efficacement à la question d'existence de relation entre les éléments:
la connexité, la complétude, l'accessibilité ... \textsf{Neo4j} est
utilisé par \textsf{Adobe, Viadeo, Deutsche Telecom, box,
etc}\cite{neo4jClients}. 
Il est important de noter que \textsf{Neo4j} répond aux enjeux traditionnels 
d'une \textsf{BDD}, principalement l'\textsf{ACIDité} des transactions. 
Il est considéré comme \textsf{NoSQL} juste parce qu'il n'utilise
pas le schéma relationnel classique.
\\
\\
\textsf{BigTable}: \index{Bigtable} Solution \textsf{NoSQL} orientée 
\textsf{colonnes} propriétaire, développé et exploité par Google. Cette 
\textsf{BDD} est proposée via la plate-forme d'application 
\textsf{Google App Engine}. Son
développement a commencé en $2004$ et \textsf{BigTable} est aujourd'hui 
utilisé par les
applications \textsf{Google} telles que \textsf{Google Earth},
\textsf{Blogger.com}, \textsf{Google Code hosting}, \textsf{YouTube},
\textsf{Gmail}. \textsf{BigTable} a servi d'inspiration à des projets 
\textsf{Open Source} de solution \textsf{NoSQL} tels que \textsf{HBase},
\textsf{Cassandra} ou \textsf{Hypertable}\cite{RickCattell}.
\\
\\
\textit{N.B}: certains \textsf{SGBD} comme \textsf{Neo4j} sont considérés comme \textsf{NoSQL} pour la seule et simple qu'il n'utilise
pas le modèle relationnel de données représentées en tables. \textsf{Neo4j} implémente les propriétés \textsf{ACID}. Pour la suite, le terme 
\textsf{NoSQL} désignera tous les \textsf{SGBD} qui ont délibérément fait fi de l'\textsf{ACIDité} pour plus de performance et de scalabilité.  
