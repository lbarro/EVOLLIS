\documentclass[12pt]{report}
\usepackage[francais]{babel}
\usepackage{lmodern}
\usepackage[a4paper]{geometry}
\usepackage[T1]{fontenc}
\usepackage[utf8]{inputenc}  
\usepackage{moreverb}
\usepackage{amsmath}
\usepackage{amsfonts}
\usepackage{amssymb}
\usepackage{textcomp}
\usepackage{pifont}
\usepackage{geometry}
\usepackage[pdftex]{graphicx}
\usepackage{graphics}
\usepackage{url}
\usepackage{graphicx}
\usepackage{float}
\usepackage{color}
\usepackage[nottoc, notlof, notlot]{tocbibind}
\usepackage[french]{varioref}
\usepackage[Glenn]{fncychap}
\usepackage{pdfpages}

\usepackage[colorlinks=true]{hyperref}

%% \hypersetup{
%% backref=true,
%% %permet d'ajouter des liens dans...
%% pagebackref=true,%...les bibliographies
%% hyperindex=true, %ajoute des liens dans les index.
%% colorlinks=true, %colorise les liens
%% breaklinks=true, %permet le retour à la ligne dans les liens trop longs
%% urlcolor= blue, %couleur des hyperliens
%% linkcolor= blue, %couleur des liens internes
%% bookmarks=true, %créé des signets pour Acrobat
%% bookmarksopen=true,
%% %si les signets Acrobat sont créés,
%% %les afficher complètement.
%% pdftitle={Mon fabuleux livre}, %informations apparaissant dans
%% pdfauthor={Pejvan BEIGUI},
%% %dans les informations du document
%% pdfsubject={Mac OS X}
%% %sous Acrobat.
%% }

\hypersetup{urlcolor=blue,linkcolor=blue,citecolor=blue,colorlinks=true}

\usepackage{multicol}

% entête et pied de page
%% \usepackage{fancyhdr} 
%% \pagestyle{fancy}
%% \renewcommand{\chaptermark}[1]{\markboth{#1}{}}
%% \renewcommand{\sectionmark}[1]{\markright{\thesection\ #1}}
%% \fancyhf{} \fancyhead[LE,RO]{\bfseries\thepage}
%% \fancyhead[LO]{\bfseries\rightmark}
%% \fancyhead[RE]{\bfseries\leftmark}
%% \renewcommand{\headrulewidth}{0.5pt}
%% \addtolength{\headheight}{0.5pt}
%% \renewcommand{\footrulewidth}{0pt}
%% \fancypagestyle{plain}{ \fancyhead{}
%% \renewcommand{\headrulewidth}{0pt}} 

% pour inclure du code par exemple
\usepackage{listings}

\lstset{%configuration de listings
float=hbp,%
basicstyle=\ttfamily\small, %
columns=flexible, %
tabsize=2, %
frame=trBL, %
frameround=tttt, %
extendedchars=true, %
showspaces=false, %
showstringspaces=false, %
numbers=left, %
numberstyle=\tiny, %
breaklines=true, %
breakautoindent=true, %
captionpos=b,%
xrightmargin=0cm, %
xleftmargin=-0cm, %
language=tex, %
frameround=fttt;%
}

%%%%%%%%%%%%%%%% Lengths %%%%%%%%%%%%%%%%
\geometry{a4paper,twoside,left=2cm,right=2cm,marginparwidth=1.2cm,marginparsep=3mm,top=1.7cm,bottom=1.5cm}

\newcommand{\stamp}{{\tt \textit{Stamp }}}
\newcommand{\class}{{\tt \textit{class }}}
\newcommand{\initarg}{{\tt \textit{Initarg }}}
\bibliographystyle{plain}
\urlstyle{sf}

%%%%%%%%%%%%%%%%%%%%%%%%%%%%%%%%%%%%%%%%%%%%%%%

\newenvironment{vcenterpage}
{\newpage\vspace*{\fill}}
{\vspace*{\fill}\par\pagebreak}

\newtheorem{ex}{Exemple}%[section]
\newtheorem{theo}{Theorem}
\newcommand{\tuple}[1]{\ensuremath{\langle #1 \rangle}}

\begin{document}

%%%% Page de titre %%%%
\def\logo{
  \begin {figure}[H]
	\includegraphics[scale=0.33]{\DIR/img/logo.jpg}
        \hspace{1cm}
        \includegraphics[scale=0.2]{\DIR/img/logobordeaux1.jpg}
        \hspace{2.8cm}
        \includegraphics[scale=0.05]{\DIR/img/logoEvollis.png}        
        \hspace{0.8cm}   
        \includegraphics[scale=0.3]{\DIR/img/logoLaBRI.jpeg}	

	\label{logo}
  \end {figure}
}

\def\title{Mise en place d'une solution non relationnelle de gestion de données}

\def\intervenant{
        \begin{flushleft}
	  \begin{tabbing}
		\textbf{Maître de stage:}
                \hspace{7.2cm} \=\textbf{Étudiant:} \\
                \noindent Eric de {\sc Marignan}
                \> {\sc Barro} Lissy Maxime\\
                \> \\
                \noindent \textbf{Tuteur à} {\sc bordeaux}1 {\bf :}
                \> Élève ingénieur\\
                \noindent Mohamed {\sc Mosbah}
                \> Dernière année - Option GL \\ 
                \noindent Sofian {\sc Maabout} \> {\sc enseirb-matmeca} 2011/2012 \\
                \>\\
                \noindent \textbf{Tuteur à l'{\sc enseirb}:}         
                \> \\
                \noindent Denis {\sc Lapoire}
	  \end{tabbing}
        \end{flushleft}
}

\def\date{\today}%9 juin 201}

\def\job{PROJET DE FIN D'ÉTUDE / STAGE MASTER 2 RECHERCHE}

\begin{titlepage}
  \logo
  \begin{flushleft}
    \textbf{École Nationale Supérieure d’Électronique, Informatique,
      Télécommunications, Mathématique et Mécanique de Bordeaux}

    \vspace{0.5cm}

    \textsf{Département informatique}

    \vspace{0.5cm}

   1 avenue du Dr Albert Schweitzer\\
   B.P. 99 33402 Talence Cedex

    
  \end{flushleft}
  
  \vspace{4cm}
	\begin{center}
	  {\bf \job}\\
	  \vspace{1cm}
		 {\LARGE\bf \title}\\
\vspace{1cm}
           \date

	\end{center}


        \vspace{4cm}
        \intervenant
\end{titlepage}


%%%%% terminologie %%%%%%%%%%
\newpage

\def\termea{\bf \footnotesize SGBD}
\def\sensa{\footnotesize Système de Gestion de Base de Données}

\def\termeb{\bf \footnotesize SGBDR}
\def\sensb{\footnotesize Système de Gestion de Base de Données Relationnel}

\def\termec{\bf \footnotesize BDD}
\def\sensc{\footnotesize Base De Données}

\def\termec{\bf \footnotesize BDDR}
\def\sensc{\footnotesize Base De Données Relationnelle}

\def\termed{\bf \footnotesize }
\def\sensd{\footnotesize }

\def\termee{\bf \footnotesize }
\def\sense{\footnotesize }

\def\termef{\bf \footnotesize }
\def\sensf{\footnotesize }

\def\termeg{\bf \footnotesize }
\def\sensg{\footnotesize }

\def\termeh{\bf \footnotesize }
\def\sensh{\footnotesize }

\def\termei{\bf \footnotesize }
\def\sensi{\footnotesize }

\def\termej{\bf \footnotesize }
\def\sensj{\footnotesize }

\def\termek{\bf \footnotesize }
\def\sensk{\footnotesize }



\begin{center}
\subsubsection*{Terminologie}
\begin{tabular}{|p{5cm}|p{12cm}|}
\hline
{\bf ~~~~~ T{\scriptsize ERME}} &  {\bf ~~~~~~~~~~~~~~~~~~~ S{\scriptsize IGNIFICATION}}\\
\hline
\hline
\termea & \sensa\\
\hline
\termeb & \sensb\\
\hline
\termec & \sensc\\
\hline
\termed & \sensd\\
\hline
\termee & \sense\\
\hline
\termef & \sensf\\
\hline
\termeg & \sensg\\
\hline
\termeh & \sensh\\
\hline
\termei & \sensi\\
\hline
\termej & \sensj\\
\hline
\termek & \sensk\\
\hline

\end{tabular}
 
\end{center}


%%%% resume %%%%%%%%%
\begin{abstract}
  resume

\vspace{1cm}

\begin{center}
\textbf{Abstract}
\end{center}

\noindent abstract

\end{abstract}

%%%% plan %%%%%
\tableofcontents
%\listoffigures
%\listoftables

%%%% remerciement %%%%%%%
\addcontentsline{toc}{chapter}{Remerciements}
\chapter*{Remerciements}
Merci à tous!!


%%%%% corps du rapport %%%%%%%%%
\addcontentsline{toc}{chapter}{Introduction}
\chapter*{Introduction}

The article ~\cite{pd} talks about how to visualize graphs containing many nodes and edges. Improvements in data acquisition leads to an increase of the size and the complexity of graphs and this huge amount of data generally causes visual clutter, in our case due to edge crossing.
For example, it could be interesting to visualize data in fields like biology, social sciences, data mining or computer science, and then emphasize their high-level pattern to help users perceive underlying models.


Nowadays, in the research world, the information is easily represented into graphs to visualize more and more data. However, this huge amount of information prevents the graph from being manually drawn:  It explains the need of automatic methods able to generate an appropriate graph with all nodes and edges. Yet this graph may suffer from cluttering, which should be reduced for a better understanding.

Our objective all along this project is to read what has been done before relating to this problem, to provide an objective point of view on those previous works, and propose our contribution. We have implemented a method, then optimized its performances with
 current technologies (OpenMP, Tulip...) and setted our boundaries. 


The first part of this document presents review-related work on reducing edge clutters and enhancing edge bundle visualization, with which the article is connected. The second will deal with the Tutte algorithm and its differents versions. A third part will talk about the implementation issues and show our results. Finally, we draw a conclusion and explain the limits of our work for further improvements.


\part{Présentation du cadre de mon stage}

\chapter{Présentation de l'organisme d'acceuil: {\sf EVOLLIS}}
\chapter{Présentation du projet \sf XXXXX}
\section{Le contexte}
\section{L'environnement de travail}
\section{L'équipe de travail}

\part{Déroulement de mon stage}

\chapter{État de l'art}
\section{Le NoSQL}

Le NoSQL signifiant littéralement « \textsf{Not only SQL} » est une
dénomination désignant une catégorie de gestionnaire de bases de
données massives. Ces gestionnaires \textsf{NoSQL} optent pour la
simplicité au détriment de certaines fonctionnalités classiques des
\textsf{SGBD}\footnote{Système de Gestion de Base de Données}
relationnels pour plus de performance et une meilleure scalabilité.
Le \textsf{NoSQL} tente d'apporter une solution face à la montée en
charges des bases de données dans le monde de l'\textsf{Internet}. En
effet, les \textsf{SGBD} non-relationnels sont plus anciens que les
\textsf{SGBD} relationnels et sont répandus sur les mainframes et les
logiciels d'annuaire. Par exemple, le stockage d'information à l'aide
de tableaux associatifs existe depuis le début de
l'histoire des bases de données, en 1970. 
Les \textsf{SGBD} non-relationnels ont connu une nouvelle jeunesse avec le
\textsf{NoSQL} dans le domaine des services \textsf{Internet}. La
plupart des logiciels \textsf{NoSQL} sont destinés à être utilisés
dans les dispositifs en répartition de charge des grands services
\textsf{Internet}.

\subsection{Les différents types de bases de données \textsf{NoSQL}} 

\section{SQL vs NoSQL}

\section{Échange de données SQL/NoSQL}

\section{Rootage intelligent entre SQL et NoSQL}


\chapter{Choix des \textsf{NoSQL} à tester}
\section{Le \textsf{NoSQL mongoDB}}
\noindent \textbf{Norme de stockage}: Le \textsf{BSON} un dérivé binaire du \textsf{JSON} qui offre un bon rapport entre la place occupée et la rapidité de parsing notamment en intégrant la taille de chaque entrée pour pouvoir passer à la suivante si la clé ne correspond pas.
\\
\\
Toute cette partie est profondement inspirée de ... Avec quelques détails supplémentaires dont les sources sont citées 
au fur et à mesure.
\\
\\
\textsf{replicaset et shards} sont deux notions fondamentales dans la distribution des données de MongoDB.
\\
\\
Limitations dans l'indexations:
On ne peut pas indexé des données (par un exemple un document ou des champs dans un document) depassant 800 bytes.
\\
On peut indexer tant que la RAM le permet.
\\
On a pas bésoin de réorganiser les données en table en table ou de changer leur formet avant de les stocker en base. Quand on accède aux données, aucun traitement de remise en forme est necessaire. Pas besoin d'un code intermediaire. MongoDB s'occupe de tout. 
\\
Les grandes idées:
* L'organisation physique: collection, document, base de données, format de stockage BSON.
* 

\section{Le \textsf{NoSQL} \textsf{cubeLaBRI}}
\input{src/cube}

\chapter{Les tests de charge}
\section{Tests de charge sur \textsf{mongoDB}}
\section{Tests de charge sur \textsf{cubeLaBRI}}
\section{Le choix final d'\textsf{EVOLLIS} de la solution \textsf{NoSQL}}

\chapter{Pilote du \textsf{NoSQL} xxxx pour \sf  EVOLLIS}

\addcontentsline{toc}{chapter}{Conclusion}
%\chapter*{Conclusion}

After understanding problems due to the grid building, we proposed the Tutte algorithm to increase data comprehension and ease graph reading. This permits us to obtain uniformized size triangles and increase angles between edges while avoiding edge crossing. 
Finally, our method allowed an improvement of the clutter reduction and help unterstanding. Furthermore, thanks to our optimizations, the performance is comparable to existing methods.


As explained below, the Tutte algorithm only works in some specifics cases. We showed it may fail for a concave polygon or when some nodes are fixed. However, the grid used is particular: it is created by applying a quad-tree algorithm, then a Voronoi decomposition and has the following constraint: two fixed nodes can not be linked by an edge.
Unfortunately, despite these grid characteristics, we also discovered that the apply of the Tutte algorithm on this graph leads to edge crossing, which breaks the planarity of the final graph. This is because the grid used may contain some fixed nodes created during the grid building.


In future work, this problem could be solved by the uniformization of the triangle size, i.e. automatically join small triangles or divide big triangles into smaller ones. It is nevertheless possible to create a post-treatment function to fix edge crossing. 
Some others issues concerning our program remain unsolved and could be taken into account in the future. Rather than work with input graph nodes, it would be interesting to dynamically add nodes (and their edges to keep an internally triangulated planar graph) and to launch again the algorithm in order to refine the given results.



\nocite{NoSQLCmp}

%\bibliographystyle{unsrt}
\bibliography{bibliographie}

%%%%%% Liste des annexes %%%%%%%%%%
\part{Annexes}
\appendix
\chapter{Le principe \textsf{ACID}}\label{acid}



\end{document}
