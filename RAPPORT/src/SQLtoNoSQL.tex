Pour profiter des performances d'un \textsf{NoSQL}, il est essentiel
de définir à l'avance la représentation qu'on fera des
données. L'utilité et le choix du \textsf{NoSQL} doit principalement
dépendre de la représentation des données. Adopter le \textsf{NoSQL} à
cause de la volumétrie en faisant juste une traduction par équivalence
entre des \textsf{BDD} relationnelle et non-relationnelle ne garantit
pas la performance et surtout la base perd l'intégrité qu'offre le modèle 
rélationnel.  
\\ \\ 
Les « \textsf{NoSQL} » offrent juste une autre façon
de représenter les données pour faciliter les accès      et mise à jour des
données afin de gagner en performance. Pour la gestion des données 
organisée en tables, la structure des \textsf{SGBDR} y est exclusivement
consacrée. Le \textsf{NoSQL} n'est donc pas là pour remplacer systématiquement
le  \textsf{SQL}. Cependant
certaines solutions \textsf{NoSQL}, en exemple \textsf{MongoDB}, ont
exhibé une charte de correspondance entre les composants d'une
\textsf{BDDR} \footnote{Database, Table, Index, colonne, clé
  primaire ...} et leurs composants. \textsf{MongoDB} a également mis
en place une charte de traduction requête \textsf{SQL} / requête
\textsf{MongoDB}.
\\
\\
\textsf{SQL} d'une part, \textsf{NoSQL} d'autre part, l'un n'exclut pas l'autre.
Tout dépend de l'usage qu'on compte en faire ou du profit qu'on
souhaite en tirer. La bonne méthode est de les faire cohabiter et de stocker
les données dans l'un ou dans l'autre en fonction des caractéristiques de 
celles-ci. C'est ce que fait d'ailleurs \textsf{Google} qui
même après avoir mis en place sa solution
propriétaire \textsf{BigTable} utilise \textsf{MySQL}
notamment pour son programme publicitaire \textsf{AdWords}, sa
principale source de revenue et qui utilise d'énormes quantités de
données. Cependant imaginer un protocole d'échange entre \textsf{SQL}
et \textsf{NoSQL} peut s'avérer très fastidieux dans la mesure où les
deux conçoivent les données très différemment.
\\
\\
Il existe une solution \textsf{Open Source} permettant des échanges
entre \textsf{SQL}/\textsf{NoSqL}:\textsf{Sqoop} ou « SQL To Hadoop
». \textsf{Hadoop}\index{Sqoop} est un assemblage de plusieurs
sous-projet avec à la base un système de
fichier \textsf{HDFS}\footnote{Hadoop Distributed File System}
et \textsf{HBASE}\footnote{\textsf{SGBD} non-relationnelle orienté
colonnes} pour le stockage. Développé en \textsf{java}, il est destiné
aux applications distribuées et à la gestion intensive des
données. \textsf{Sqoop} charge une table ou toute la base en système
de fichiers \textsf{HDFS} en fonction des options d'importation
spécifiées et génère aussi des classes \textsf{java} correspondantes
aux tables de la bases, ainsi que des objets correspondants aux lignes
pour permettre d'interagir avec les données importées.
\\
\\
Mise à part \textsf{Scoop} qui est une solution pour \textsf{Hadoop}
d'échange avec le \textsf{SQL}, le framework
\textsf{spring} intègre des modules \textsf{Data access}\cite{springsource} permettant 
de persister des objets \textsf{POJO} dans des \textsf{BDD NoSQL}. En exemple, 
le module \textsf{Spring Data MongoDB} pour persister dans \textsf{MongoDB}. Avec \textsf{spring} il est alors possible
d'effectuer une échange en deux temps:
\begin{enumerate}
\item D'abord transformer les données dans la \textsf{BDDR} en des objets \textsf{POJO} par le biais de module déjà existant comme \textsf{hibernate, toplink, iBatis}...
\item Puis persister les objets \textsf{POJO} dans par le biais du module \textsf{Data access} dédie.    
\end{enumerate} 
Cependant il faut changer de module \textsf{Data access} à chaque fois
qu'il faut changer de \textsf{BDD NoSQL}. Il est difficile de mettre
en place une standardisation des interfaces dans la mesure où
chaque \textsf{BDD NoSQL} vient avec son propre langage de requête.
