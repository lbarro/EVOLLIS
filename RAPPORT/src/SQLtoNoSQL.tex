Pour profiter des performances d'un \textsf{NoSQL}, il est essentiel
de définir à l'avance la représentation qu'on fera des
données. L'utilité et le choix du \textsf{NoSQL} doit principalement
dépendre de la représentation des données. Adopter le \textsf{NoSQL} à
cause de la volumétrie en faisant juste une traduction par équivalence
entre des \textsf{BDDs} relationnelle et non-relationnelle ne garantit
pas la performance et surtout on risque de perdre en intégrité des
données. Et cette pensée de \textsf{Ted Dziuba} illustre bien le risque que
courent les entreprises qui migrent d'une solution \textsf{MySQL} vers  une 
solution \textsf{NoSQL}: ceci reviendrait à « {\sf échanger une liste de limitations 
et de bugs connus pour [...] une liste de limitations et de bugs inconnus} »\cite{DieNosql}.
\\ \\ Les « \textsf{NoSQL} » offrent juste une autre façon
de représenter les données pour faciliter les accès et mise à jour des
données afin de gagner en performance. Ceux-ci n'offrent pas forcement
une meilleure performance quand les données sont représentées de la
même manière. À priori imaginer un échange naïf entre \textsf{SQL} et
\textsf{NoSQL} de reproduction à l'identique sans modification de la
représentation n'est d'aucun intérêt pour la performance. Cependant
certaine solution \textsf{NoSQL} à l'exemple de \textsf{MongoDB} ont
exhibé une charte de correspondance entre les composants d'une
\textsf{BDD} relationnel\footnote{Database, Table, Index, colonne, clé
  primaire ...} et leurs composants. \textsf{MongoDB} a également mis
en place une charte de traduction requête \textsf{SQL} / requête
\textsf{MongoDB}\footnote{Pour plus de détails
  \url{http://www.mongodb.org/display/DOCS/SQL+to+Mongo+Mapping+Chart}}.
\\
\\
\textsf{SQL} d'une part, \textsf{NoSQL} d'autre part, l'un n'exclut pas l'autre
tout dépend de l'usage qu'on compte en faire ou du profit qu'on
souhaite en tirer. L'idéal serait de les faire cohabiter et de stocker
les données dans l'un ou dans l'autre en fonction de leurs
caractéristiques. C'est ce que fait d'ailleurs \textsf{Google} qui
même après avoir mis en place sa solution
propriétaire \textsf{BigTable} utilise les \textsf{BDD} relationnelles
notamment pour son programme publicitaire \textsf{AdWords}, sa
principale source de revenue et qui utilise d'énormes quantités de
données. Cependant imaginer un protocole d'échange entre \textsf{SQL}
et \textsf{NoSQL} peut s'avérer très fastidieux dans la mesure où les
deux ont des représentations de données très différentes.
\\
\\
Il existe une solution \textsf{Open Source} permettant des échanges entre \textsf{SQL} et \textsf{NoSqL}:\textsf{Sqoop} ou « SQL To Hadoop ». \textsf{Hadoop}\index{Sqoop} est un assemblage de plusieurs sous-projet avec à 
la base  un système de fichier \textsf{HDFS}\footnote{Hadoop Distributed File System} et \textsf{HBASE}\footnote{\textsf{SGBD} non-relationnelle orienté colonnes} pour le stockage. Développé en \textsf{java}, il est destiné aux applications distribuées et à la gestion intensive des données. \textsf{Sqoop} charge une table ou toute la base en système de fichiers \textsf{HDFS} en fonction des options d'importation spécifiées et génère aussi des classes \textsf{java} correspondantes aux tables de la bases, ainsi que des objets correspondants aux lignes pour permettre d'interagir avec les données importées.
\\
\\
Enfin, chaque base de données NoSQL vient avec son propre langage de
requête, rendant difficile la standardisation des interfaces entre les
applications\cite{SergeLeblal}.
