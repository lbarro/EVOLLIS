\textsf{NoSQL} ne veut pas dire « \textsf{No SQL} » mais plutôt «
  \textsf{Not Only SQL} ». On pourrait penser que la mouvance
\textsf{NoSQL} soit là pour remplacer les \textsf{BDD}
relationnelles classiques. Mais ceci est difficilement imaginable de
nos jours. L’intérêt pour le \textsf{NoSQL} s'est considérablement
accru dans le monde des entreprises à l'issu des annonces d’adoption 
de ces technologies par les grands acteurs d'\textsf{Internet} tels 
\textsf{Google} et \textsf{Facebook} qui se sont multipliées. Son bon 
fonctionnement dans des conditions extrêmes lui confère une certaine fiabilité.
\\
\\
Ci-dessous quelques traitements auxquels le modèle relationnel classique ne répond pas
et qui pourrait motiver le recours aux technologies \textsf{NoSQL}\index{Limites du \textsf{SQL}}:

\paragraph{Indexation d'une quantité de documents:}
Un problème se pose quant à l'indexation d'une quantité importante de données. Par exemple pour le \textsf{SGBD} \textsf{SQL Server}, la version $2008$ supporte au plus $999$ indexes et chaque index, chaque index peut couvrir au maximum 16 colonnes et la somme des tailles des colonnes couvertes ne doit pas excéder $900\ octets$\cite{SQLserver} 

\paragraph{Sites à fort trafic:} Pour faire face à des volumes importants de données et y avoir accès à de différents endroit0s du monde, il faut pouvoir répliquer ces données sur différentes machines physiques et le \textsf{SGBD} relationnel classique montre des limites dans un environnement distribué. 

\paragraph{Données de taille variables selon les enregistrements:} Les \textsf{BDD} relationnelles classiques prévoient un schéma statique à l'avance où dans une table donnée, les lignes sont identiques en type et en nombre de colonnes donc offre un environnement non dynamique pour les enregistrements.

\paragraph{Réécritures fréquentes:} le modèle classique prévoit plus de lectures que d'écritures.

\paragraph{Extensibilité de la base:} Comme signaler plus haut, \textsf{Cassandra} qui est un \textsf{NoSQL} offre la possibilité d'ajouter facilement des serveurs pour gagner en performance sans modifier la structure de la base. Les \textsf{SGBD} relationnels classiques n'offrent pas cette possibilité.
\\ \\ 
Sans doute, l'extensibilité requise et la grande quantité de données et de mises à jour rendent le modèle relationnel inefficace ce qui a obligé à trouver un nouveau modèle. 
Face à ces encouragements en l'occurrence ceux énumérés ci-dessus, on serait tenter
d'adopter sans équivoque le \textsf{NoSQL}. Cependant il est important
de prendre ne considération quelques aspects au risque d'une mauvaise
utilisation. « L’intérêt d’une base de données \textsf{NoSQL} pour un
  projet ne dépend pas du volume de données qu’elle aura à
  manipuler. Le choix de son utilisation doit être basé sur la
  préférence d’un mode de représentation et non sur une forte
  volumétrie »\cite{NoSQLeurope}. Aussi, Ne devrions-nous pas oublier
que cette catégorie de produits, fait le compromis d'abandonner certaines fonctionnalités classiques
des \textsf{SGBD} relationnels au profit de la performance. Par
conséquent, il ne s’agit donc pas d’une solution miracle pour tout
type de stockage de données.  La tentative de reproduire dans une base
de données \textsf{NoSQL} une représentation ou un comportement
habituellement offert par un \textsf{SGBD} relationnels aboutira
probablement à une solution peu efficace.
