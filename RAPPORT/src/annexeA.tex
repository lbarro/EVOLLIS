\chapter{Les propriétés \textsf{ACID}}\label{acid}
Les propriétés \textsf{ACID} permettent à un \textsf{SGBD} d'effectuer
des transactions. Par transaction, il faut comprendre une suite
d'opérations qui font passer la \textsf{BDD} d'un état antérieur à un
état postérieur. Les états intermédiaires entre les états avant la
transaction et après la transaction ne sont pas visibles. Ses
propriétés sont les suivantes:

\paragraph{Atomicité:} la suite d'opérations constituant une transaction 
est indivisible. La transaction est entièrement effectuée ou pas du tout. 
Il y a annulation de toute la transaction lorsqu'une des opérations échoue.
S'il est question de modifier une série de valeurs et qu'une modification
échoue alors toutes les valeurs déjà modifiées reprennent leurs anciennes
valeurs. 

\paragraph{Cohérence:} quelque soit l'opération effectuée, la base doit garder
un état cohérent. Toute transaction qui viole par exemple une règle d'intégrité 
échoue. Après la fusion de deux tables, les entrées doivent toutes 
avoir des identités différentes. Si ce n'est pas le cas alors la fusion n'est 
pas effectuée.

\paragraph{Isolation:} chaque transaction est isolée de sorte à ce qu'elle est 
seule peut voir les modifications pendant son exécution. Toute
transaction enclenchée en parallèle d'une autre voit la version des
données antérieure à celle-ci.  Il existe 4 niveaux d'isolation
définis dans le standard \textsf{ANSI/ISO SQL}:
\begin{enumerate}
\item \textsf{Uncommited read ou lectures des données non validées}.
  Ce niveau est le niveau d'isolation le plus léger. Avec un tel
  niveau d'isolation le système se comporte comme s'il n'y en avait
  pas. Les modifications apportées par une transaction non validée
  sont visibles.

\item \textsf{Commited read ou lecture des seules données
  validées}. Les modifications lors d'une transaction ne sont visibles
  que lorsqu'elle termine.  Cependant lors d'une transaction une
  donnée peut changer sans que la transaction en cours en soit
  responsable. Ceci peut arriver, par exemple, lorsqu'une transaction
  en parallèle termine et que ses modifications sont validées.

\item \textsf{Repeatable read ou lecture répétée}. Ce niveau
  fonctionne comme le niveau précédent à la seule différence que
  durant son exécution, une transaction ne voit pas les mises à jour
  effectuées par d'éventuelles transactions qui se sont exécutées en
  parallèle. D'où « \textsf{repeatable read} » pour souligner que pendant une
  transaction, une donnée aura toujours la même valeur en
  lecture si elle n'est pas modifiée par la transaction elle-même.
  Cependant tout nouveau rajout validé de données au système par une
  transaction qui a terminé est visible par toute autre transaction en
  cours.

\item \textsf{Serializable ou sérialisable}. Ce niveau est le niveau
  d'isolation le plus poussé. Le système se comporte comme s'il n'y
  avait qu'une transaction à la fois. Pendant son exécution une
  transaction ne voit ni les mises à jour, ni les rajouts de données
  au système des autres transactions. D'où « \textsf{serializable} » pour
  mettre en relief le caractère d'exécution en série des transactions
  plutôt qu'en parallèle.
\end{enumerate}

\paragraph{Durabilité:} dès lors qu'une transaction est validée, aucune défaillance
du système ne pourra conduire à l'annulation de celle-ci. Les modifications liées à
une transaction validée perdurent et ne sont jamais remises en cause.
