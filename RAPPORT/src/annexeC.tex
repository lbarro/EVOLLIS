\chapter{Le théorème de \textsf{Brewer} ou le théorème \textsf{CAP}}\label{cap}
Jusqu'en $2002$, le théorème n'était qu'une conjecture de énoncée par \textsf{Brewer}. 
\newtheorem{theorem}{Théorème}[section]
\begin{theorem}
\emph{(de Brewer)}
  Dans un environnement distribué, un système ne peut être doté qu'au plus de deux des trois propriétés suivantes:
\begin{dinglist}{70}
  \item Cohérence (\textsf{C}onsistency)
  \item Disponibilité (\textsf{A}vailability)
  \item Résistance au morcellement (\textsf{P}artition Tolerance)
\end{dinglist}
\end{theorem}
\noindent
\textsf{Cohérence: } toute opération effectuée sur un nœud est automatiquement visible sur tous les autres nœuds du système. Si j'écris sur un noeud et lis sur un autre, je dois    
\textsf{Disponibilité: }
\textsf{Résistance au morcellement: }
\newpage
