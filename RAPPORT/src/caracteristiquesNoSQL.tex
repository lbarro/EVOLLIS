Originally motivated by Web 2.0 applications, these
systems are designed to scale to thousands or millions
of users doing updates as well as reads, in contrast to
traditional DBMSs and data warehouses.

These systems typically sacrifice some of
these dimensions, e.g. database-wide transaction
consistency, in order to achieve others, e.g. higher
availability and scalability

Here are the six key features of a \textsf{NoSQL}:

1. the ability to horizontally scale “simple
operation” throughput over many servers,

2. the ability to replicate and to distribute (partition)
data over many servers,

3. a simple call level interface or protocol (in
contrast to a SQL binding),

4. a weaker concurrency model than the ACID
transactions of most relational (SQL) database
systems,

5. efficient use of distributed indexes and RAM for
data storage, and

6. the ability to dynamically add new attributes to
data records

The NoSQL systems described here generally do not
provide ACID transactional properties: updates are
eventually propagated, but there are limited guarantees
on the consistency of reads. Some authors suggest a
“BASE” acronym in contrast to the “ACID” acronym:
• BASE = Basically Available, Soft state,
Eventually consistent
• ACID = Atomicity, Consistency, Isolation, and
Durability
The idea is that by giving up ACID constraints, one
can achieve much higher performance and scalability\\
==================================================\\
ACID et les NoSQL en introduisant BASE\\
CAP et les NoSQL\\

**************************************************
Dans un prémier temps parler des points commun entre
les nosql

Maintenant je vais aborder la notion du théorème de brewer encore appelé
le théorème de CAP en rapport avec les solutions \textsf{NoSQL}. 
\textsf{Brewer} dit que dans un environnement distribué 
une base de données ne paut avoir cohérent, disponible et resistant
au morcellement. Ceci n'est qu'un petit rappelle sinon ce théorème 
plus en détail à l'annexe \ref{cap} du présent document.  
