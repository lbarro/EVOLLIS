Dans la mouvance \textsf{NoSQL}, les données sont représentées de
diverses manières. Les gestionnaires \textsf{NoSQL} sont ainsi classés
par catégorie en fonction de la représentation des données. Ci-dessous
les 4 principales catégories:
\\
\\
\textsf{Clé - valeur}\index{NoSQL!Orienté clé - valeur}: 
Représentation la plus simple, limitée à un mapping entre un ensemble
de clé et un ensemble de valeurs. À chaque clé est associée une seule
valeur dont elle ne connaît pas la structure. Ce postulat s'applique
aux cas d'utilisation où les lectures et écritures sont réduites à un
accès disque simple, c'est à dire que le système se charge uniquement
de stocker et de restituer sans effectuer des opérations, ni sur la
structuration des valeurs stockées, ni sur les relations entre
elles. Cette catégorie trouve sa légitimité dans le constat que les
applications accèdent à la base de données pour les mêmes
informations. Les bases << clé-valeur » fonctionnent généralement en
parallèle avec une autre base considérée comme la principale. Elles
vont servir à stocker les résultats des requêtes récurrentes pour
éviter de répéter les mêmes opérations dans la base
principale\cite{cleValeur}.
\\
\\ 
{\sf Document}: Ajoute\index{NoSQL!Orienté document} au
modèle \textsf{clé-valeur} une valeur à structure non plane qui
nécessiterait un ensemble de jointures en logique relationnelle. La
valeur est sous la forme d'un document contenant des données
organisées de manière hiérarchique, à l’image de ce que permettent
\textsf{XML} ou \textsf{JSON}\footnote{Java Script Object Notation}. 
La principale différence avec le modèle \textsf{clé-valeur} réside
dans le fait que les valeurs stockée sont désormais structurées. Cette
propriété permet un accès beaucoup plus élaboré, non limité à une
lecture ou écriture par clé, à la base de données. Cependant des
opérations d'\textit{hoverhead} supplémentaires notamment pour
vérifier la structuration des valeurs stockées, diminuent sa
performance par rapport au modèle \textsf{clé-valeur} en terme de
lecture et écriture des données.
\\
\\ 
{\sf Colonne}\index{NoSQL!Orienté Colonne}: Autre évolution du modèle
clé-valeur, il permet de disposer d'un très grand nombre de valeurs
sur une même ligne, permettant ainsi de stocker les relations de type
one-to-many. Les lignes peuvent avoir des types de colonnes différents
et également des nombres de colonnes différents. Il y a également une
hiérarchie entre les colonnes. Cette configuration permet d’effectuer
des requêtes par clé ou des opérations d'ensemble par enregistrement.
Ceci n'est
pas réalisable avec le modèle \textsf{clé-valeur} ou le
modèle \textsf{document} dans la mesure où il y a une seule valeur par
clé.
\\
\\
{\sf Graphe}\index{NoSQL!Orienté graphe}: très adapté à la
modélisation, au stockage et à la manipulation des relations
non-triviales ou variables entre les données. À l'image de la gestion
des liens d'amitié sur les réseaux sociaux comme \textsf{Facebook}. 
Ces informations sont difficilement
modélisables dans une base de données relationnelle.
\\
\\ 
Le site \url{http://nosql-database.org} recense $122$
solutions \textsf{NoSQL} réparties en $9$ catégories dont les
principales sont celles citées ci-dessus. Pour une
illustration de celles-ci sur un exemple concret, se référer à l'annexe \ref{data}.
