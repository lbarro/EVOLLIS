Dans la mouvance \textsf{NoSQL}, les données sont représentées de
diverses manières. Les gestionnaires \textsf{NoSQL} sont ainsi classés par catégorie en fonction de la représentation des données. Ci-dessous les 4 principales catégories: 
\\\\ \textsf{Clé - valeur}\index{NoSQL!Orienté clé - valeur}: Représentation la plus simple limitée à un simple mapping entre un ensemble de clé et un ensemble de valeur. À chaque clé est associée une seule valeur dont elle ne connaît pas la structure. Cette structure est très adaptée à la
gestion de caches ou pour fournir un accès rapide aux
informations. Ce postulat
permet en général d’atteindre des performances bien supérieures dans
la mesure où les lectures et écritures sont réduites à un accès disque
simple. Cette catégorie trouve sa légitimité dans le constat que les
applications présentent de nombreux accès à la base de données qui ne
sont que de simples lectures ou écritures à partir d’un
identifiant\cite{cleValeur}.  \\\\ {\sf Document}:
Ajoute\index{NoSQL!Orienté document} au modèle clé-valeur,
l’association d’une valeur à structure non plane, c’est-à-dire qui
nécessiterait un ensemble de jointures en logique relationnelle.  la
valeur est sous la forme d'un document contenant des données
organisées de manière hiérarchique à l’image de ce que permettent
\textsf{XML} ou \textsf{JSON}.  \\\\ {\sf Colonne}\index{NoSQL!Orienté
  Colonne}: Autre évolution du modèle clé-valeur, il permet de
disposer d'un très grand nombre de valeurs sur une même ligne,
permettant ainsi de stocker les relations de type one-to-many. Les
lignes peuvent avoir des types de colonnes différents et également des
nombres de colonnes différents. On dispose également d'une hiérarchie
entre les colonnes. Contrairement au système Clé-Valeur, celui-ci
permet d’effectuer des requêtes par clé.  \\\\ {\sf
  Graphe}\index{NoSQL!Orienté graphe}: très adapté pour la modélisation, le
stockage et la manipulation des relations non-triviales ou variables
entre les données. Un cas classique d'utilisation de cette catégorie
est le stockage des informations des réseaux sociaux. Ces informations
sont difficilement modélisables dans une base de données
relationnelle.  \\ \\ {\color{red} {\sf La représentation utilisée par
    cubeLaBRI}: encore inconnu} \\ \\ \textsf{D'autres
  représentations}: \textsf{Xml}, \textsf{bases de données objet},
\textsf{les bases de données hiérarchiques ou encore les datagrids}
...
