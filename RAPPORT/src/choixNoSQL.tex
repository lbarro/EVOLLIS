Comme mentionné dans le contexte à la section \ref{contexte}, \textsf{Evollis} utilise actuellement une
\textsf{BDDR} classique, \textsf{PGSQL}. Il utilise également un système de cache, le \textsf{Memcached}. Dans la section \ref{membase} du présent document, j'ai aussi parlé de \textsf{Membase} qui peut être vue comme une amélioration
de \textsf{Memcached}, en ce sens qu'il est compatible avec toutes les applications \textsf{Memcached} avec des fonctionnalités
supplémentaires telles que la persistance, le réplication et le clustering.  
Avec le système clé-valeur simple comme \textsf{Memcached}, le système n'est pas conscient de la structure de la valeur. 
La partie persistance de \textsf{Membase} est orientée document et restant dans la catégorie des \textsf{BDD} orientées document,  
\textsf{Evollis} veut tester une autre solution \textsf{NoSQL} qui est \textsf{MongoDB}. Il utilise le mode de stockage clé-valeur
mais rend le système conscient de la structure de la valeur stockée dans le but de permettre un champ
opérationnelle plus large sur la base de données. Parmi les nombreuses solutions de la famille \textsf{NoSQL} des \textsf{BDD}, le choix des solutions à étudiées s'est limité aux solutions \textsf{Memcached} et \textsf{MongoDB}.
