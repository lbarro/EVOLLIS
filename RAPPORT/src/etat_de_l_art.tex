\section{Le NoSQL}

Le NoSQL signifiant littéralement « \textsf{Not only SQL} » est une
dénomination désignant une catégorie de gestionnaire de bases de
données massives. Ces gestionnaires \textsf{NoSQL} optent pour la
simplicité au détriment de certaines fonctionnalités classiques des
\textsf{SGBD}\footnote{Système de Gestion de Base de Données}
relationnels pour plus de performance et une meilleure scalabilité.
Le \textsf{NoSQL} tente d'apporter une solution face à la montée en
charges des bases de données dans le monde de l'\textsf{Internet}. En
effet, les \textsf{SGBD} non-relationnels sont plus anciens que les
\textsf{SGBD} relationnels et sont répandus sur les mainframes et les
logiciels d'annuaire. Par exemple, le stockage d'information à l'aide
de tableaux associatifs existe depuis le début de
l'histoire des bases de données, en 1970. 
Les \textsf{SGBD} non-relationnels ont connu une nouvelle jeunesse avec le
\textsf{NoSQL} dans le domaine des services \textsf{Internet}. La
plupart des logiciels \textsf{NoSQL} sont destinés à être utilisés
dans les dispositifs en répartition de charge des grands services
\textsf{Internet}.

\subsection{Les différents types de bases de données \textsf{NoSQL}} 

\section{SQL vs NoSQL}

\section{Échange de données SQL/NoSQL}

\section{Rootage intelligent entre SQL et NoSQL}
