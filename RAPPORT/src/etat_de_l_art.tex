Avant de rentrer dans le vif du sujet, il est fondamental dans un
premier temps de s'imprégner des différentes notions qui le détermine.
Ainsi dans l'état de l'art vais-je aborder la notion clé du contexte de
mon stage qui est le \textsf{NoSQL}. Il s'agit en effet d'une nouvelle
famille de gestionnaire de \textsf{BDD}. Il en existe plusieurs réparties 
par catégories. Je prendrai le soin de parler de ces différentes
catégories avec des exemples à l'appui. 
Je termine l'étude sur la notion \textsf{NoSQL} en définissant clairement les
caractéristiques des gestionnaires \textsf{NoSQL} qui sont en quelque
sorte un ensemble de propriétés qu'ils ont en commun. D'un premier
abord, le \textsf{NoSQL} est considéré comme une alternative
au \textsf{SQL}. Je ferai donc un rapprochement de ces deux
familles en mettant en évidence ce qui les distingue. Ensuite je termine
sur l'étude en abordant la problématique d'échange entre 
les deux familles
\textsf{SQL} et \textsf{NoSQL}.

