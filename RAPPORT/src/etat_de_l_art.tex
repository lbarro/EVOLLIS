\section{Le \textsf{NoSQL}}

Le NoSQL signifiant littéralement « \textsf{Not only SQL} » est une
dénomination désignant une catégorie de gestionnaire de bases de
données massives. Ces gestionnaires \textsf{NoSQL} optent pour la
simplicité au détriment de certaines fonctionnalités classiques des
\textsf{SGBD}\footnote{Système de Gestion de Base de Données}
relationnels pour plus de performance et une meilleure
scalabilité. Cependant, notons que les \textsf{SGBD} non-relationnels
sont plus anciens que les \textsf{SGBD} relationnels et sont répandus
sur les mainframes et les logiciels d'annuaire. Par exemple, le
stockage d'information à l'aide de tableaux associatifs existe depuis
le début de l'histoire des bases de données, en 1970.  \\ \\ Les
\textsf{SGBD} non-relationnels ont connu une nouvelle jeunesse avec le
\textsf{NoSQL} dans le domaine des services \textsf{Internet}. Ce
retour aux non-relationnels est essentiellement motivé par les
nouvelles demandes en rapport avec les sites web de grande audience
apparus dans les années 2000. La plupart des logiciels \textsf{NoSQL}
sont destinés à être utilisés dans les dispositifs en répartition de
charge des grands services \textsf{Internet}. Ceci dit que le
\textsf{NoSQL} a pour ambition d'apporter une solution face à la
montée en charges des bases de données dans le monde de
l'\textsf{Internet}.  \\ \\ La conférence meet-up\index{Meet-up 2009 à
  San-Francisco} de 2009 à \textsf{San-Francisco} est considérée comme
l'inauguration de la communauté des développeurs de logiciels
\textsf{NoSQL}. Cependant et comme le souligne \textsf{wikipedia}, les
leaders de la communauté \textsf{NoSQL} sont des \textsf{start-up} de
développeurs critiqués de ne pas avoir les moyens d'acheter les
\textsf{SGBD} de Oracle\cite{wikiNoSQL}. Il est important de noter que
la communauté \textsf{NoSQL} a effectivement mis en place des produits
capables de manipuler de très grandes quantités de données qui se
mesurent en centaines de \texttt{Téraoctets} et offrent une meilleure
scalabilité mais force est de remarquer que les solutions
\textsf{NoSQL} sont seulement adaptées pour certains besoins comme
ceux des applications \texttt{Web 2.0}.  \\\\ On reproche à la
mouvance \textsf{NoSQL} de sacrifier le principe
\textsf{ACID}\footnote{voir annexe \ref{acid}} donc de ne pas pouvoir
garantir une grande sûreté dans l'accès aux données. Cependant une
nouvelle mouvance, le \textsf{NewSQL}, ten,te de conserver la
structure classique en colonnes tout en faisant appel à différents
procédés dans le but de conserver la rapidité même sur de larges
volumes\cite{newSQL}.

\subsection{Les différents types de bases de données \textsf{NoSQL}} 
 
Dans la mouvance \textsf{NoSQL}, les données sont représentées de
diverses manières. Ainsi définit-on des catégories de gestionnaire
\textsf{NoSQL} en fonction de la manière de représenter les données.
\\\\ \textsf{Clé - valeur}\index{NoSQL!Orienté clé - valeur}: La
représentation la plus simple. Cette structure est très adaptée à la
gestion de caches ou pour fournir un accès rapide aux
informations. Comme pour une table de hashage, chaque clé est associée
à une seule valeur dont elle ne connaît pas la structure. Ce postulat
permet en général d’atteindre des performances bien supérieures dans
la mesure où les lectures et écritures sont réduites à un accès disque
simple. Cette catégorie trouve sa légitimité dans le constat que les
applications présentent de nombreux accès à la base de données qui ne
sont que de simples lectures ou écritures à partir d’un
identifiant\cite{cleValeur}.  \\\\ {\sf Document}:
Ajoute\index{NoSQL!Orienté document} au modèle clé-valeur,
l’association d’une valeur à structure non plane, c’est-à-dire qui
nécessiterait un ensemble de jointures en logique relationnelle.  la
valeur est sous la forme d'un document contenant des données
organisées de manière hiérarchique à l’image de ce que permettent
\textsf{XML} ou \textsf{JSON}.  \\\\ {\sf Colonne}\index{NoSQL!Orienté
  Colonne}: Autre évolution du modèle clé-valeur, il permet de
disposer d'un très grand nombre de valeurs sur une même ligne,
permettant ainsi de stocker les relations de type one-to-many. Les
lignes peuvent avoir des types de colonnes différents et également des
nombres de colonnes différents. On dispose également d'une hiérarchie
entre les colonnes. Contrairement au système Clé-Valeur, celui-ci
permet d’effectuer des requêtes par clé.  \\\\ {\sf
  Graphe}\index{NoSQL!Orienté graphe}: très à la modélisation, le
stockage et la manipulation des relations non-triviales ou variables
entre les données. Un cas classique d'utilisation de cette catégorie
est le stockage des informations des réseaux sociaux. Ces informations
sont difficilement modélisables dans une base de données
relationnelle.  \\ \\ {\color{red} {\sf La représentation utilisée par
    cubeLaBRI}: encore inconnu} \\ \\ \textsf{D'autres
  représentations}: \textsf{Xml}, \textsf{bases de données objet},
\textsf{les bases de données hiérarchiques ou encore les datagrids}
...

\subsection{Quelques exemples de \textsf{NoSQL}}

Les logiciels n'utilisant pas le modèle relationnel classique est en
nombre croissant et les exemples de \textsf{NoSQL} correspondent aux
\textsf{NoSQL} en open source les plus connus exceptés
\textsf{BigTable} et \textsf{cubeLaBRI} qui sont propriétaires.
\\\\ {\sf MongoDB}: MongoDB est une base de données « \textsf{orientée
    document} » totalement open-source développée en
\textsf{C++}. MongoDB stocke les données sous forme de \textsf{JSON}
dont la structure reste libre et dynamique.  En effet, aucun schéma de
\textsf{BDD}\footnote{Base de données} à respecter n'est défini à
l'avance.\cite{mongoDB} \\\\ \textsf{Cassandra}: \index{Cassandra} une
solution NoSQL d’Apache.  C’est une base de données orientée colonnes
qui se veut être hautement extensible. C’est-à-dire qu’il est très
simple d’ajouter ou d’enlever - c’est rare - un nœud du
cluster. D’abord développé par Facebook, le code source est devenu
Open-Source et repris par la fondation Apache\cite{cassandra}.  La
technologie est encore peu maîtrisée mais elles revienne régulièrement
dans l’actualité du fait des régulières annonces de migrations de
quelques entreprises web renommée\cite{cassandra2}.
\\\\ \textsf{BigTable}: \index{Bigtable} système de gestion de base de
données propriétaire, développé et exploité par Google qui ne
distribue pas sa base de données mais propose une utilisation publique
via sa plateforme d'application \textsf{Google App Engine}. Son
développement a commencé en 2004 et est aujourd'hui utilisé par les
applications Google telles que \textsf{Google Earth},
\textsf{Blogger.com}, \textsf{Google Code hosting}, \textsf{YouTube},
\textsf{Gmail} ... D'autres projets libres tels que \textsf{HBase},
\textsf{Cassandra} ou \textsf{Hypertable} s'en sont
inspirés.\cite{bigtable} \\\\ {\color{red} \textsf{cubeLaBRI}: en
  attente de documentation.}

\section{\textsf{SQL} vs \textsf{NoSQL}}

\section{Échange entre \textsf{SQL} et \textsf{NoSQL}}

\section{Système de Rootage intelligent entre {\sf SQL} et {\sf NoSQL}}
