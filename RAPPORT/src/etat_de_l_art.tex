\section{Le \textsf{NoSQL}}

Le NoSQL signifiant littéralement « \textsf{Not only SQL} » est une
dénomination désignant une catégorie de gestionnaire de bases de
données massives. Ces gestionnaires \textsf{NoSQL} optent pour la
simplicité au détriment de certaines fonctionnalités classiques des
\textsf{SGBD}\footnote{Système de Gestion de Base de Données}
relationnels pour plus de performance et une meilleure
scalabilité. Cependant, notons que les \textsf{SGBD} non-relationnels
sont plus anciens que les \textsf{SGBD} relationnels et sont répandus
sur les mainframes et les logiciels d'annuaire. Par exemple, le
stockage d'information à l'aide de tableaux associatifs existe depuis
le début de l'histoire des bases de données, en 1970.  \\ \\ Les
\textsf{SGBD} non-relationnels ont connu une nouvelle jeunesse avec le
\textsf{NoSQL} dans le domaine des services \textsf{Internet}. Ce
retour aux non-relationnels est essentiellement motivé par les
nouvelles demandes en rapport avec les sites web de grande audience
apparus dans les années 2000. La plupart des logiciels \textsf{NoSQL}
sont destinés à être utilisés dans les dispositifs en répartition de
charge des grands services \textsf{Internet}. Ceci dit que le
\textsf{NoSQL} a pour ambition d'apporter une solution face à la
montée en charges des bases de données dans le monde de
l'\textsf{Internet}.  \\ \\ La conférence meet-up de 2009 à
\textsf{San-Francisco} est considérée comme l'inauguration de la
communauté des développeurs de logiciels \textsf{NoSQL}. Cependant et
comme le souligne \textsf{wikipedia}, les leaders de la communauté
NoSQL sont des \textsf{start-up} de développeurs critiqués de ne pas
avoir les moyens d'acheter les SGBD de Oracle\cite{wikiNoSQL}. Il est
important de noter que la communauté \textsf{NoSQL} a effectivement
mis en place des produits capables de manipuler de très grandes
quantités de données qui se mesurent en centaines de
\texttt{Téraoctets} et offrent une meilleure scalabilité mais force
est de remarquer que les solutions \textsf{NoSQL} sont seulement
adaptées pour certains besoins comme ceux des applications \texttt{Web
  2.0}.  \\\\ On reproche à la mouvance \textsf{NoSQL} de sacrifier le
principe \textsf{ACID}\footnote{voir annexe \ref{acid}} donc de ne
pas pouvoir garantir une grande sûreté dans l'accès aux
données. Cependant une nouvelle mouvance, le \textsf{NewSQL}, ten,te
de conserver la structure classique en colonnes tout en faisant appel
à différents procédés dans le but de conserver la rapidité même sur de
larges volumes\cite{newSQL}.

\subsection{Les différents types de bases de données \textsf{NoSQL}} 
 
Dans la mouvance \textsf{NoSQL}, les données sont représentées de
diverses manières. Ainsi définit-on des catégories de gestionnaire
\textsf{NoSQL} en fonction de la manière de représenter les données.
\\\\ 
\textsf{Clé - valeur}: La représentation la plus simple. Cette
structure est très adaptée à la gestion de caches ou pour fournir un
accès rapide aux informations. Elle fonctionne comme un grand tableau
associatif et retourne une valeur dont elle ne connaît pas la
structure.  
\\\\ 
{\sf Document}: Ajoute au modèle clé-valeur,
l’association d’une valeur à structure non plane, c’est-à-dire qui
nécessiterait un ensemble de jointures en logique relationnelle.
\\\\ 
{\sf Colonne}: Autre évolution du modèle clé-valeur, il permet de
disposer d'un très grand nombre de valeurs sur une même ligne,
permettant ainsi de stocker les relations de type one-to-many.
Contrairement au système Clé-Valeur, celui-ci permet d’effectuer des
requêtes par clé.  
\\\\ 
{\sf Graphe}: Permet la modélisation, le
stockage et la manipulation de données complexes liées par des
relations non-triviales ou variables.  
\\ \\ 
{\sf La représentation utilisée par cubeLaBRI}: encore inconnu 
\\ \\ 
\textsf{D'autres représentations}: \textsf{Xml}, \textsf{bases de données objet},
\textsf{les bases de données hiérarchiques ou encore les datagrids}
...

\subsection{Quelques exemples de \textsf{NoSQL}}

Les logiciels n'utilisant pas le modèle relationnel classique est en
nombre croissant et les exemples de \textsf{NoSQL} correspondent aux
\textsf{NoSQL} en open source les plus connus excepté le dernier
exemple \textsf{cubeLaBRI}.  
\\\\ 
{\sf MongoDB}:
\\\\ 
\textsf{BigTable}: 
\\\\ 
\textsf{Cassandra}: une solution NoSQL d’Apache. 
C’est une base de données orientée colonnes
qui se veut être hautement extensible. C’est-à-dire qu’il est très
simple d’ajouter ou d’enlever - c’est rare - un nœud du
cluster. D’abord développé par Facebook, le code source est devenu
Open-Source et repris par la fondation Apache\cite{cassandra}.
\\\\ 
\textsf{cubeLaBRI}: en attente de documentation.

\section{\textsf{SQL} vs \textsf{NoSQL}}

\section{Échange entre \textsf{SQL} et \textsf{NoSQL}}

\section{Système de Rootage intelligent entre {\sf SQL} et {\sf NoSQL}}
