Avant de rentrer dans le vif d'un sujet, il est fondamental dans un
premier temps de s'imprégner des différentes notions qui le détermine.
Ainsi dans l'état de l'art aborderai-je la notion clé du contexte de
mon stage qui est le \textsf{NoSQL}. Il s'agit en effet d'une nouvelle
famille de gestionnaire de \textsf{BDD}. Il en existe plusieurs types
et je prendrai le soin d'en parler avec des exemples à l'appui. Je
termine l'étude la notion \textsf{NoSQL} en exprimant clairement les
caractéristiques des gestionnaires \textsf{NoSQL} qui sont en quelque
sorte un ensemble de propriétés qu'ils ont en commun. Le \textsf{SQL}
dans le \textsf{NoSQL} désigne une autre famille de gestionnaires.
Ils sont les plus utilisés dans la gestion des données. D'un premier
abord, le \textsf{NoSQL} est conçu comme une alternative
au \textsf{SQL}. Je ferai donc un rapprochement entre ces deux
familles en mettant en évidence ce qui les distingue. Ensuite je termine
cet état de l'art en abordant la problématique d'échange entre les deux familles
\textsf{SQL} et \textsf{NoSQL}.

