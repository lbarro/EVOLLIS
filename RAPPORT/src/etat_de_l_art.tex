\section{Le NoSQL}

Le NoSQL signifiant littéralement « \textsf{Not only SQL} » est une
dénomination désignant une catégorie de gestionnaire de bases de
données massives. Ces gestionnaires \textsf{NoSQL} optent pour la
simplicité au détriment de certaines fonctionnalités classiques des
\textsf{SGBD}\footnote{Système de Gestion de Base de Données}
relationnels pour plus de performance et une meilleure
scalabilité.\\\\ Le \textsf{NoSQL} tente d'apporter une solution face
à la montée en charges des bases de données dans le monde de
l'\textsf{Internet}. En effet, les \textsf{SGBD} non-relationnels sont
plus anciens que les \textsf{SGBD} relationnels et sont répandus sur
les mainframes et les logiciels d'annuaire. Par exemple, le stockage
d'information à l'aide de tableaux associatifs existe depuis le début
de l'histoire des bases de données, en 1970.  \\ \\ Les \textsf{SGBD}
non-relationnels ont connu une nouvelle jeunesse avec le
\textsf{NoSQL} dans le domaine des services \textsf{Internet}. Ce
retour aux non-relationnels est essentiellement motivé par les
nouvelles demandes en rapport avec les sites web de grande audience
apparus dans les années 2000. La plupart des logiciels \textsf{NoSQL}
sont destinés à être utilisés dans les dispositifs en répartition de
charge des grands services \textsf{Internet}. La conférence meet-up de
2009 à \textsf{San-Francisco} est considérée comme l'inauguration de
la communauté des développeurs de logiciels \textsf{NoSQL}.\\\\ Pour
reprendre les mots du site
\textsf{wikipedia}\footnote{\url{http://fr.wikipedia.org/wiki/NoSQL}}:«
  Les leaders de cette communauté sont des start-up de développeurs
  qui n'ont pas les moyens d'acheter les SGBD de Oracle et qui ont
  développé leurs propres SGBD en imitant les produits de Google et
  Amazon.com. Les produits qu'ils ont créés sont capables de manipuler
  les très grandes quantités de données (qui se mesurent en centaines
  de Téraoctets) et offrent une scalabilité bien adaptée au besoins
  des applications Web 2.0 ce qui les rend attractifs sur le marché
  des SGBD. Les auteurs décrivent leurs produits comme n'étant pas des
  SGBD, mais plutôt des logiciels de stockage de données »
 
\subsection{Les différents types de bases de données \textsf{NoSQL}} 

\subsection{Quelques exemples de \textsf{NoSQL}}

\section{SQL vs NoSQL}

\section{Échange de données SQL/NoSQL}

\section{Rootage intelligent entre SQL et NoSQL}
