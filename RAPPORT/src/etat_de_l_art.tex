\section{Le NoSQL}

Le NoSQL signifiant littéralement « \textsf{Not only SQL} » est une
dénomination désignant une catégorie de gestionnaire de bases de
données massives. Ces gestionnaires \textsf{NoSQL} optent pour la
simplicité au détriment de certaines fonctionnalités classiques des
\textsf{SGBD}\footnote{Système de Gestion de Base de Données}
relationnels pour plus de performance et une meilleure
scalabilité. Cependant, notons que les \textsf{SGBD} non-relationnels
sont plus anciens que les \textsf{SGBD} relationnels et sont répandus
sur les mainframes et les logiciels d'annuaire. Par exemple, le
stockage d'information à l'aide de tableaux associatifs existe depuis
le début de l'histoire des bases de données, en 1970.  \\ \\ Les
\textsf{SGBD} non-relationnels ont connu une nouvelle jeunesse avec le
\textsf{NoSQL} dans le domaine des services \textsf{Internet}. Ce
retour aux non-relationnels est essentiellement motivé par les
nouvelles demandes en rapport avec les sites web de grande audience
apparus dans les années 2000. La plupart des logiciels \textsf{NoSQL}
sont destinés à être utilisés dans les dispositifs en répartition de
charge des grands services \textsf{Internet}. Ceci dit le
\textsf{NoSQL} a pour ambition d'apporter une solution face à la
montée en charges des bases de données dans le monde de
l'\textsf{Internet}.  \\ \\ La conférence meet-up de 2009 à
\textsf{San-Francisco} est considérée comme l'inauguration de la
communauté des développeurs de logiciels \textsf{NoSQL}. Cependant et
comme le souligne \textsf{wikipedia}, les leaders de la communauté
NoSQL sont des \textsf{start-up} de développeurs critiqués de ne pas
avoir les moyens d'acheter les SGBD de Oracle\cite{wikiNoSQL}. Il est
important de noter que la communauté \textsf{NoSQL} a effectivement
mis en place des produits capables de manipuler de très grandes
quantités de données qui se mesurent en centaines de
\texttt{Téraoctets} et offrent une meilleure scalabilité mais force
est de remarquer que les solutions \textsf{NoSQL} sont seulement
adaptées pour certains besoins comme ceux des applications \texttt{Web
  2.0}.  \\\\ On reproche à la mouvance \textsf{NoSQL} de sacrifier le
principe \textsf{ACID}\footnote{voir l'annexe \ref{acid}} donc de ne
pas pouvoir garantir une grande sûreté dans l'accès aux
données. Cependant une nouvelle mouvance, le \textsf{NewSQL}, ten,te
de conserver la structure classique en colonnes tout en faisant appel
à différents procédés dans le but de conserver la rapidité même sur de
larges volumes\cite{newSQL}.

\subsection{Les différents types de bases de données \textsf{NoSQL}} 

\subsection{Quelques exemples de \textsf{NoSQL}}

\section{SQL vs NoSQL}

\section{Échange de données SQL/NoSQL}

\section{Rootage intelligent entre SQL et NoSQL}
