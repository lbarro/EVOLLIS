Comme mentionné à la section précédente, le
site \url{http://nosql-database.org} recense $122$
solutions \textsf{NoSQL}.  J'en présenterai seulement 5. Un exemple
par catégorie énumérée précédemment. Je m'intéresserai
particulièrement aux solutions « \textsf{Open Source} » très
utilisées. Je parlerai aussi de la solution propriétaire \textsf{BigTable}
qui est considérée comme la première de la
mouvance \textsf{NoSQL}. Dans l'ordre, \textsf{Membase} une solution
orientée \textsf{clé-valeur}, \textsf{MongoDB} une solution
orientée \textsf{document}, \textsf{cassandra} une solution
orientée \textsf{colonne}, \textsf{Néo4j} une solution
orientée \textsf{graphe} et \textsf{BigTable} une solution
propriétaire orientée \textsf{colonne}.
\\
\\
\textsf{Membase}: \textsf{Membase} Solution \textsf{NoSQL} orientée \textsf{clé-valeur}, Open Source et diffusée sous la licence Apache 2.0. Écrit en \textsf{C++/Erlang}, il est soutenu par l'entreprise du même nom \textsf{Membase}\cite{RickCattell}. Il a été développé par les leaders du projet \textsf{Memcached} qui est un
autre système de stockage \textsf{clé-valeur}. Ce dernier permet la
gestion de mémoire cache distribuée, utilise la \textsf{RAM} et n'est
donc pas persistant. \textsf{Memcached} fut à l'origine développé
par \textsf{Brad Fitzpatrick} pour le \textsf{LiveJournal} en $2003$
et est aujourd'hui utilisé part de nombreux sites
tels \textsf{Wikipedia, Flickr, Bebo, Twitter, Typepad, Yellowbot,
Youtube, Digg, WordPress.com, Craigslist,
Mixi} \footnote{\url{http://memcached.org/about}}. \textsf{Membase}
rajoute au système \textsf{Memcached} la propriété de persistance sur
le disque, la réplication des données pour assurer la résistance aux
pannes et le clustering. Il est compatible avec les applications
memcached existantes.
\\
\\ 
{\sf MongoDB}: \index{MongoDB} Solution \textsf{NoSQL} orientée \textsf{document},
Open Source et diffusée sous la licence \textsf{GPL}. Écrit
en \textsf{C++}, il est soutenu
par \textsf{10gen}\cite{RickCattell}. Son développement a débuté en Octobre $2007$ et sa prémière version public est sortie en Février $2009$\footnote{\url{http://blog.mongodb.org/post/434865639/state-of-mongodb-march-2010}}. Sur son
site \url{http://www.10gen.com/what-is-mongodb}, \textsf{10gen} écrit
que le but de \textsf{MongoDB} est de combiner le
modèle \textsf{clé-valeur}, rapide et scalable, et le modèle
relationnel qui permet des opérations complexes comme les jointures et
l'indexation. \textsf{MongoDB} stocke les données sous le format
binaire \textsf{BSON} de \textsf{JSON}\footnote{Ces notions sont
expliquées plus en détail à la section \ref{mongoDB} du présent
document} dont la structure reste libre et dynamique.  En effet, aucun
schéma de \textsf{BDD} à respecter n'est prédéfini\cite{mongoDB}. Le
site web «\textsf{ChinaVisual}», la plus grande média en ligne
chinoise\footnote{D'après \url{http://www.aboutus.org/ChinaVisual.com}},
a annoncé en début d'année $2009$ sa migration de \textsf{MySQL}
vers \textsf{MongoDB}\cite{GUYunhua} alors que \textsf{10gen} venait juste de publier la prémière version stable. Aujourd'hui, \textsf{MongoDB} est utilisé par de grands acteurs de la toile comme \textsf{Buddy Media, Craigslist, Disney, Forbes, foursquare, Intuit, MTV Networks, Shutterfly, Traackr, Wordnik}\footnote{Le site \url{http://www.10gen.com/customers} donne des explications sur les motivations de ses différents clients}.
\\
\\ 
\textsf{Cassandra}: \index{Cassandra} Solution \textsf{NoSQL} orientée \textsf{colonnes} d’\textsf{Apache}.  C’est une \textsf{BDD}
orientée colonnes qui se veut être hautement extensible. C’est-à-dire
qu’il est très simple d’ajouter ou d’enlever - c’est rare - un nœud du
cluster. D’abord développé par \textsf{Facebook}, le code source est
devenu « \textsf{Open-Source} » et repris par la
fondation \textsf{Apache}\cite{cassandra}.  La technologie est encore
peu maîtrisée mais elles revienne régulièrement dans l’actualité du
fait des régulières annonces de migrations de quelques
entreprises \textsf{web} renommées\cite{cassandra2}.
\\
\\
\textsf{Néo4j}: \index{Néo4j}
\\
\\
\textsf{BigTable}: \index{Bigtable} système de gestion de base de
données propriétaire, développé et exploité par Google qui ne
distribue pas sa base de données mais propose une utilisation publique
via sa plateforme d'application \textsf{Google App Engine}. Son
développement a commencé en $2004$ et est aujourd'hui utilisé par les
applications \textsf{Google} telles que \textsf{Google Earth},
\textsf{Blogger.com}, \textsf{Google Code hosting}, \textsf{YouTube},
\textsf{Gmail} ... D'autres projets libres tels que \textsf{HBase},
\textsf{Cassandra} ou \textsf{Hypertable} s'en sont
inspirés.
