Les logiciels n'utilisant pas le modèle relationnel classique est en
nombre croissant et les exemples de \textsf{NoSQL} ci-dessous correspondent à ceux en open source, assez connus exceptés
\textsf{BigTable} et \textsf{cubeLaBRI} qui sont propriétaires.
\\\\ {\sf MongoDB}: MongoDB est une base de données « \textsf{orientée
    document} » totalement open-source développée en
\textsf{C++}. MongoDB stocke les données sous forme de \textsf{JSON}
dont la structure reste libre et dynamique.  En effet, aucun schéma de
\textsf{BDD} à respecter n'est défini à
l'avance.\cite{mongoDB} \\\\ \textsf{Cassandra}: \index{Cassandra} une
solution \textsf{NoSQL} d’\textsf{Apache}.  C’est une \textsf{BDD} orientée colonnes
qui se veut être hautement extensible. C’est-à-dire qu’il est très
simple d’ajouter ou d’enlever - c’est rare - un nœud du
cluster. D’abord développé par \textsf{Facebook}, le code source est devenu
« \textsf{Open-Source} » et repris par la fondation \textsf{Apache}\cite{cassandra}.  La
technologie est encore peu maîtrisée mais elles revienne régulièrement
dans l’actualité du fait des régulières annonces de migrations de
quelques entreprises \textsf{web} renommées\cite{cassandra2}.
\\\\ \textsf{BigTable}: \index{Bigtable} système de gestion de base de
données propriétaire, développé et exploité par Google qui ne
distribue pas sa base de données mais propose une utilisation publique
via sa plateforme d'application \textsf{Google App Engine}. Son
développement a commencé en $2004$ et est aujourd'hui utilisé par les
applications \textsf{Google} telles que \textsf{Google Earth},
\textsf{Blogger.com}, \textsf{Google Code hosting}, \textsf{YouTube},
\textsf{Gmail} ... D'autres projets libres tels que \textsf{HBase},
\textsf{Cassandra} ou \textsf{Hypertable} s'en sont
inspirés.\\\\ {\color{red} \textsf{cubeLaBRI}: en
  attente de documentation.}
