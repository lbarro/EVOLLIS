Sur le site \url{nosql-database.org}, sont recensés $122$
solutions \textsf{NoSQL}.  J'en présenterai seulement 5. Un exemple
par catégorie. Lesquelles catégories qui sont au nombre de 4, ont été
citées à la section \ref{categorie}. Je m'intéresserai
particulièrement aux solutions « \textsf{Open Source} » très
utilisées. Je parlerai aussi d'une solution
propriétaire \textsf{BigTable} qui est considérée comme la première de
la mouvance \textsf{NoSQL}. Dans l'ordre, je parlerai de \textsf{Riak} une solution \textsf{clé-valeur}, de \textsf{MongoDB} une solution \textsf{document}, de \textsf{cassandra} une solution \textsf{colonne}, de \textsf{Néo4j} une solution \textsf{graphe} et enfin de la solution \textsf{BigTable}.
\\
\\ 
{\sf MongoDB}: MongoDB est une base de données « \textsf{orientée
    document} » totalement open-source développée en
\textsf{C++}. MongoDB stocke les données sous forme de \textsf{JSON}
dont la structure reste libre et dynamique.  En effet, aucun schéma de
\textsf{BDD} à respecter n'est défini à
l'avance.\cite{mongoDB} 
\\
\\ 
\textsf{Cassandra}: \index{Cassandra} une
solution \textsf{NoSQL} d’\textsf{Apache}.  C’est une \textsf{BDD}
orientée colonnes qui se veut être hautement extensible. C’est-à-dire
qu’il est très simple d’ajouter ou d’enlever - c’est rare - un nœud du
cluster. D’abord développé par \textsf{Facebook}, le code source est
devenu « \textsf{Open-Source} » et repris par la
fondation \textsf{Apache}\cite{cassandra}.  La technologie est encore
peu maîtrisée mais elles revienne régulièrement dans l’actualité du
fait des régulières annonces de migrations de quelques
entreprises \textsf{web} renommées\cite{cassandra2}.
\\
\\ 
\textsf{BigTable}: \index{Bigtable} système de gestion de base de
données propriétaire, développé et exploité par Google qui ne
distribue pas sa base de données mais propose une utilisation publique
via sa plateforme d'application \textsf{Google App Engine}. Son
développement a commencé en $2004$ et est aujourd'hui utilisé par les
applications \textsf{Google} telles que \textsf{Google Earth},
\textsf{Blogger.com}, \textsf{Google Code hosting}, \textsf{YouTube},
\textsf{Gmail} ... D'autres projets libres tels que \textsf{HBase},
\textsf{Cassandra} ou \textsf{Hypertable} s'en sont
inspirés.
