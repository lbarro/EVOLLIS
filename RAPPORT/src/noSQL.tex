Le \textsf{NoSQL} signifiant littéralement « \textsf{Not only SQL} »
est une dénomination désignant une nouvelle catégorie de gestionnaires
de bases de données massives. Le \textsf{NoSQL} est une nouvelle
mouvance dans la gestion des données qui se départit du relationnel 
pour rechercher plus de performance et de scalabilité. Dans une
base de données relationnelle, les données sont structurées dans des
tables à deux dimensions selon un modèle permettant de dresser une
relation entre elles basée sur la théorie des ensembles et la logique
mathématique. Les \textsf{BDDR} classiques ont toujours cherché à
implémenter les propriétés \textsf{ACID}\footnote{voir annexe
  \ref{acid}}\index{ACID} qui sont nécessaires à la gestion des
transactions.  
\\ 
\\ 
Pour certains cas d'utilisations comme la gestion des réseaux sociaux,
un gestionnaire n'a pas forcement besoin des
propriétés \textsf{ACID}. Le
\textsf{MySQL} qui est un \textsf{SGBDR} non transactionnel en est la
preuve par sa popularité. Il est proposé par plus de la moitié des
hébergeurs \textsf{web}. Les utilisateurs de \textsf{BDD} n'ont pas
forcement pour intention de gérer des transactions. Pour cette raison,
les gestionnaires \textsf{NoSQL} optent pour la simplicité, la
performance et la scalabilité au détriment des propriétés
\textsf{ACID} et du modèle relationnel pour mieux parer la montée en
charge dans des situations où les propriétés transactionnelles ne sont
pas exigées. Celles-ci sont à l'origine d'\textit{overhead}, temps
consacré par un système à se gérer lui-même plutôt que d'effectuer le
travail proprement dit. D'après \textsf{Michael Stonebraker}, pionnier
des \textsf{SGBD} et selon des travaux menés dans le laboratoire
de \textsf{MIT}, << 96\% [des cycles machines de \textsf{MySQL} est] de
l’\textit{overhead} » partagé entre la gestion de \textsf{buffer}, les
verrous au niveau enregistrement, l’écriture des \textsf{logs} et
le \textsf{multi-threading}. Donc seulement << 4\% [...] consacré au
travail en lui-même»
\footnote{\url{http://www.slideshare.net/Dataversity/newsql-vs-nosql-for-new-oltp-michael-stonebraker-voltdb}}.
\\ 
\\ 
Les \textsf{SGBD} non-relationnels existent depuis
le début de l'histoire des bases de données dans les années $60$
notamment avec les systèmes de gestion de fichiers plus ou moins
sophistiqués. Ils sont de nos jours rependus sur les mainframes et les
logiciels d'annuaire. Il sont plus anciens que les \textsf{SGBD}
relationnels qui ont quant à eux fait leur apparition à partir de 1970. Les
\textsf{SGBD} non-relationnels ont connu une nouvelle jeunesse avec la
mouvance \textsf{NoSQL}. La conférence meet-up de 2009
à \textsf{San-Francisco} est considérée comme l'inauguration de la
communauté des développeurs de logiciels \textsf{NoSQL}. Ce retour aux
non-relationnels est essentiellement motivé par les nouvelles demandes
de performance et de scalabilité face à la montée en charge des sites
web de grande audience apparus à partir des années $2000$. La plupart des
logiciels \textsf{NoSQL} sont ainsi destinés à être utilisés dans les
dispositifs en répartition de charge des services \textsf{Internet}.
\\ 
\\ 
Les leaders de la communauté \textsf{NoSQL} sont issus
des \textsf{start-up} internet pour lesquels la simplicité et le
gratuité des \textsf{SGBD} sont un critère de choix important quitte à
risquer l'intégrité des données. Il est important de noter que la
communauté
\textsf{NoSQL} a effectivement mis en place des produits capables de
manipuler de très grandes quantités de données qui se mesurent en
centaines de \texttt{Téraoctets} et offrent une meilleure scalabilité
mais force est de remarquer que les solutions \textsf{NoSQL} sont
seulement adaptées pour certains besoins comme ceux des applications
\texttt{Web 2.0} qui ne nécessitent pas de manoeuvres critiques comme
les transactions dans la gestion des données.  En effet le \textsf{Web
2.0} désigne l'ensemble des fonctionnalités nouvelles et usages
nouveaux du \textsf{World Wide Web} permettant aux internautes ayant
peu de connaissances techniques d’être actif sur la toile, à l’image
des réseaux sociaux.
\\
\\ 
Il y a une autre mouvance plus récente, le
\textsf{NewSQL}\index{NewSQL}, qui contrairement au \textsf{NoSQL}
tente de conserver la structure classique relationnelle tout en
faisant appel à différents procédés dans le but de conserver la
rapidité, même sur de larges volumes\cite{newSQL}. Ces derniers ne
font pas l'objet du présent document.
