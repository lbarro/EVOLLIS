Le seul dénominateur commun entre les solutions \textsf{NoSQL} est la
non utilisation du modèle relationnel
classique\cite{NealLeavitt}. Chaque solution a sa propre
représentation des données et surtout son propre langage de
requête\cite{SergeLeblal}. Cependant toute les
solutions \textsf{NoSQL} reposent sur les mêmes principes. Ci-dessous,
6 principales propriétés recherchées dans la
mouvance \textsf{NoSQL}:

\begin{enumerate}

\item \textsf{Open Source}: la majeur partie des solutions \textsf{NoSQL} est \textsf{Open Source}

\item \textsf{Scalabilité horizontale}: fonctionner sur plusieurs machines peu coûteuses plutôt que sur une seule machine puissante mais coûteuse\cite{NealLeavitt}. En d'autres termes, pour gagner en performance, il suffit de rajouter une machine plutôt que d'augmenter la puissance d'une machine. Cette propriété est surtout implémentée pour les 
      opérations simples. Par ``opération simple'', il faut comprendre
      la recherche par clé, lecture et écriture d'un enregistrement ou
      d'un petit groupe d'enregistrements. À l'opposé d'opérations
      complexes comme les jointures dans le modèle
      relationnel\cite{RickCattell}.

\item \textsf{Réplication et distribution des données}: les données sont répliquées et distribuées entre plusieurs serveurs pour améliorer l'accès et éviter les pertes de données en cas de panne. Une grande partie des solutions \textsf{NoSQL} effectue une mise à jour asynchrone. Cette notion est abordée plus en détail plus bas.

\item \textsf{Un protocole simplifié}: les solutions \textsf{NoSQL} mettent en avant la facilité d'usage, notamment une installation et configuration faciles et un langage de requête bas niveau.

\item \textsf{Une gestion de concurrence faible}: pour gagner en performance, les solutions \textsf{NoSQL} se focalisent moins sur l'intégrité des données et par conséquent sur la gestion des accès concourant aux données. Les solution \textsf{NoSQL} ne gèrent pas l'\textsf{ACIDité} des transactions. Cette est abordée plus en détail plus bas.

\item \textsf{Souplesse et dynamisme}: les \textsf{BDD NoSQL} se veulent libres de schémas et dynamiques. Ajout et suppression de colonnes sans arrêter le serveur et adaptation à n'importe quel type de données. Elles se démarquent du modèle relationnel stockent uniquement les données qui peuvent être distillées en tables.
\end{enumerate}
Après avoir survolé les valeurs défendues par la
mouvance \textsf{NoSQL}, je vais maintenant aborder la notion
d'\textsf{ACIDité} en rapport avec le \textsf{NoSQL}. Je l'ai tantôt
dit, les \textsf{NoSQL} n'ont pas pour vocation
d'être \textsf{ACID}. En rappel, les propriétés \textsf{ACID}
garantissent que la \textsf{BDD} garde toujours un état
cohérent\footnote{voir annexe \ref{acid} pour plus de
détails}. Cependant, à défaut d'être \textsf{ACID}, les
solutions \textsf{NoSQL} revendiquent
d'être \textsf{BASE}\cite{RickCattell}, acronyme de \textsf{Basically
Available, Soft state, Eventually consistency}. Avec un \textsf{BDD
BASE}, il n'y a pas la garantie que les données soient à jour sur tous
les nœuds du cluster d'où le « \textsf{Basically Available}
». Cependant, après un temps assez long où il n'y a pas de nouvelles
mises à jour d'où le « \textsf{Soft state} », la \textsf{BDD} pourra
atteindre son état cohérent d'où l'« \textsf{Eventually consistency}
».
\\
\\
Le \textsf{NoSQL} a donc abandonné la « consistance forte » pour de la
« consistance éventuelle ». Il a pris pour alibi le
théorème \textsf{CAP}\cite{MichaelStonebraker2} qui postule que dans un environnement distribué,
une \textsf{BDD} ne peut être à la fois consistante, disponible et
résistante au morcellement. Elle ne peut avoir que 2 de ces 3
propriétés. Ce théorème est détaillé à l'annexe \ref{cap} du présent
document. Ce théorème laisse entendre donc
que le \textsf{NoSQL}, en renonçant à la « consistance forte », pourra
préserver la \textsf{disponibilité} et la \textsf{résistance au
morcellement} dans un environnement distribué.
\\
\\
Ce choix est tout à fait justifié au regard des valeurs défendues par
la mouvance. Les solutions \textsf{NoSQL} entendent fonctionner dans
un environnement distribué et utilisent la méthode de réplication des
données pour résister aux pannes logicielles et machines. Elles ont pour
vocation d'apporter plus de performance que les \textsf{SGBDR} dans
l'exécution des opérations simples. Elles ont donc intérêt a choisir
parmis les trois propriétés, la \textsf{disponibilité} et
la \textsf{résistance au morcellement}.
