Partitionnement horizontale: consiste à partitionner une table par ligne.
Les lignes sont stockées dans de tables différentes. Une vue de l'unionde toutes
les partitions permet d'avoir une vue d'ensemble de toutes les données. 

Partitionnement verticale: consiste à partitionner une table par colonne.
Généralement on stocke les colonnes qui sont statics dans une table et les
colonnes qui changent dans une autre. La création d'une vue pour la table en 
entier diminue la performance mais des analyses statics sur les colonnes
se feront très rapidement.
